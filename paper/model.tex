\section{Disc-planet models}\label{model}
The system is a two-dimensional (2D) non-self-gravitating gas disc orbiting
a central star of mass $M_*$. We adopt cylindrical
co-ordinates $(r,\phi,z)$ centred on the star. The frame is   
non-rotating. Computational units are such that 
$G=M_*=\mathcal{R}=\mu=1$ where $G$ is the gravitational constant,
$\mathcal{R}$ is the gas constant and $\mu$ is the mean molecular
weight. 

The disc evolution is governed by the standard fluid equations  
\begin{align}\label{3d_gov_eq}
  &\frac{\p\Sigma}{\p t}+\nabla\cdot(\Sigma \bv)=0, \\
  & \frac{\p\bv}{\p t}+\bv\cdot\nabla\bv= -\frac{1}{\rho}\nabla p 
  - \nabla{\Phi} + \bm{f}_\nu,\\
  & \frac{\p e}{\p t} + \nabla\cdot(e\bv) = -p\nabla\cdot\bv +
  \mathcal{H} - \mathcal{C}, %\\
%  &\nabla^2\Phi_d = 4\pi G \Sigma \delta(z),\label{poisson}
\end{align}
where $\Sigma$ is the surface density, $\bv = (v_r,v_\phi)$ the fluid
velocity, $p$ is the vertically-integrated pressure, $e=p/(\gamma-1)$ is the energy
per unit area and the adiabatic index $\gamma=1.4$ is assumed
constant. 

The total potential $\Phi$ includes the stellar potential, planet potential
(described below) 
and indirect potentials to account for the non-inertial reference
frame. 
In the momentum equations, $\bm{f}_\nu$ represent viscous forces, 
which includes artificial bulk viscosity to handle shocks, and a
Navier-Stokes viscosity whose magnitude is  
characterized by a constant kinematic viscosity parameter
$\nu$. However, we will be considering effectively inviscid discs by
adopting small values of $\nu$.  

\subsection{Heating and cooling}
In the energy equation, the heating term $\mathcal{H}$ is defined as 
\begin{align}
  \mathcal{H} \equiv Q^+ - Q^+_i\frac{\Sigma}{\Sigma_i}, 
\end{align}
where $Q^+$ represents viscous heating (from both physical and
  artificial viscosity) and subscript $i$ denotes
evaluation at $t=0$. The cooling term $\mathcal{C}$ is defined as
\begin{align}
  \mathcal{C} \equiv \frac{1}{t_c}\left(e -
  e_i\frac{\Sigma}{\Sigma_i}\right),  
\end{align}
where $t_c = \beta\Omega_k^{-1}$ is the cooling time,
$\Omega_k=\sqrt{GM/r^3}$ is the Keplerian frequency and $\beta$ is an
input parameter. This cooling prescription allows one 
to explore the full range of thermodynamic response of the disc in a 
systematic way: $\beta\ll1$ is a locally isothermal disc while
$\beta\gg1$ is an adiabatic disc.  


Note that the energy source terms
have been chosen to be absent at $t=0$, allowing the disc to be
initialized close to steady state. The $\mathcal{C}$ function attempts
to restore the initial energy density (and 
therefore temperature) profile. In practice, this is a cooling term at
the gap edge because disc-planet interaction leads to heating.  

\subsection{Disc model and initial condition}
The disc occupies $r\in[r_\mathrm{in}, r_\mathrm{out}]$ and
$\phi\in[0,2\pi]$. The initial disc is axisymmetric with  
surface density profile  
 
\begin{align}\label{initial_density}
   \Sigma(r) = \Sigma_\mathrm{ref}\left(\frac{r}{r_\mathrm{in}}\right)^{-s}
    \left[1 - \sqrt{\frac{r_\mathrm{in}}{r + H_i(\rin)}}\,\right] 
\end{align}
where the power-law index $s=2$, $H(r) = \ciso\Omega_k $ defines the disc scale-height 
where $\ciso=\sqrt{p/\Sigma}$ is the isothermal sound-speed. The disc aspect ratio is defined as $h\equiv H/r$ and initially
$h=0.05$. For a non-self-gravitating disc, the surface density scale
$\Sigma_\mathrm{ref}$ is arbitrary. 

The initial azimuthal velocity $v_{\phi i}$ is set by centrifugal balance with
pressure forces and stellar gravity. For a thin disc, 
$v_{\phi}\simeq r\Omega_k$. The initial radial velocity is
$v_{r}=3\nu/r$, where $\nu = \hat{\nu}\rin^2\Omega_k(\rin)$, and we
adopt $\hat{\nu}= 10^{-9}$, so that $|v_{r}/v_{\phi}|\ll1$ and the initial 
flow is effectively only in the azimuthal direction.  With this value 
of physical viscosity, the only source of heating is through
compression, shock-heating (via artificial viscosity) and the
$\mathcal{C}$ function when $e/\Sigma<e_i/\Sigma_i$. 

\subsection{Planet potential}\label{planet_config}
The planet potential is given by 
\begin{align}
  \Phi_p = -\frac{GM_p}{\sqrt{|\bm{r} - \bm{r}_p|^2 + \epsilon_p^2}},
\end{align}
where $M_p$ is the planet mass and we fix $q\equiv M_p/M_*=10^{-3}$
throughout this work. This corresponds to a Jupiter-mass planet if $M_*=M_{\sun}$. 
The planet's position in the disc 
$\bm{r}_p=(r_p,\phi_p)$  and $\epsilon_p=0.5r_h$ is a softening
length with $r_h=(q/3)^{1/3}r_p$ being the Hill radius.  
%Hill radius chosen so eps doesn't change with time (scale-height does)
The planet is held on a fixed circular orbit with $ r_p = 10\rin$ and $\phi_p=\Omega_k(r_p)t$. 
This also
defines the time unit $P_0\equiv 2\pi/\Omega_k(r_p)$ used to describe results. 
%In this study only one planet mass is considered, with $q=10^{-3}$. 
%The initial orbital radius is 


\section{Numerical experiments}\label{method}
The disc-planet system is evolved using the 
\texttt{FARGO-ADSG} code \citep{baruteau08, baruteau08b}. This is a modified version 
of the original \texttt{FARGO} code \citep{masset00a} to include the energy 
equation. The code employs a finite-difference scheme similar 
to the \texttt{ZEUS} code \citep{stone92}, but with a modified azimuthal transport 
algorithm to circumvent the time-step restriction set by the fast rotation speed at the 
inner disc boundary. 
The disc is divided into $(N_r,N_\phi)$ zones in the radial and azimuthal directions, 
respectively. The grid spacing is logarithmic in radius and uniform in azimuth.

\subsection{Cooling prescription}
In this work we only vary one control parameter: the cooling
time. %The disc parameters are $Q_o$, which characterizes the strength of self-gravity; 
%and $\beta$ which sets the cooling time. Two levels of self-gravity is considered: 
%a light disc with $Q_o=8$ and a heavy disc with $Q_o=1.5$. 
The cooling parameter $\beta$ is chosen indirectly  through the parameter
$\tilde{\beta}$ such that 

\begin{align}
  t_c(r_p+x_s) = \beta\Omega_k^{-1}(r_p+x_s) = \tilde{\beta} t_{\mathrm{lib}}(r_p+x_s), 
\end{align}
where $x_s$ is the distance from the planet to its
gap edge, and $t_\mathrm{lib}$ is the time interval between successive
encounters of a fluid element at the gap edge and the planet's
azimuth. That is, we measure the cooling time in units of the time
interval between encounters of a fluid element at the gap edge and the
planet-induced shock. 

Assuming Keplerian orbital frequencies and $x_s\ll r_p$
gives $t_\mathrm{lib}\simeq 4\pi r_p/(3\Omega_{kp} x_s)$, where
$\Omega_{kp} = \Omega_k(r_p)$. Therefore   
\begin{align}\label{betatilde}
  \beta = \tilde{\beta} \frac{4\pi r_p}{3x_s} \left(1  - \frac{3x_s}{2r_p}\right), 
\end{align}
where $x_s\ll r_p$ was used again. We use $x_s = 2r_h$ in
Eq. \ref{betatilde}. For a planet mass with $q=10^{-3}$,
Eq. \ref{betatilde} then gives $\beta \simeq 23.9\tilde{\beta}$. In
terms of planetary orbital periods, this is
\begin{align} 
  t_c(r) = \frac{\beta}{2\pi}\left(\frac{r}{r_p}\right)^{3/2}P_0\simeq
  3.8 \tilde{\beta}\left(\frac{r}{r_p}\right)^{3/2}P_0. 
\end{align}

\subsection{Diagnostic measures}

\subsubsection{Generalised potential vorticity}

The generalised potential vorticity is defined as
\begin{align}
  \tilde{\eta} = \frac{\kappa^2}{2\Omega\Sigma}\times S^{-2/\gamma}, 
\end{align}
where $\kappa^2 = r^{-3}\partial_r(r^4\Omega^2)$ is the square of the
epicyclic frequency, $\Omega=v_\phi/r$ is the angular speed, and
$S\equiv p/\Sigma^\gamma$ is the entropy. The first factor is the
usual potential vorticity (PV, or vortensity). 

The generalised PV appears in the description of the linear stability
of radially-structured adiabatic discs \citep{lovelace99,li00}, where
the authors show an extremum in $\teta$ may lead to a dynamical
instability, the RWI. In a barotropic disc where $p=p(\Sigma)$, the entropy factor is 
absent and the important quantity is the PV. 

\subsubsection{Fourier modes} 
The RWI is characterized by exponentially
growing perturbations. Though in this paper we do not consider a
formal linear instability calculation, modal analysis will be useful
to analyse the growth of perturbations with different azimuthal
wavenumbers, which is associated with the number of vortices initially
formed by the RWI.    

The Fourier transform of the time-dependent surface density is
\begin{align}\label{fouriertransform}
  \Sigma_m(r,t) = \int_{0}^{2\pi}
  \Sigma(r,\phi,t) \, \mathrm{e}^{-\mathrm{i}m\phi} \, \mathrm{d}\phi 
\end{align} 
where $m$ is the azimuthal wave number. We define the growth rate
$\sigma$ of the $m^\mathrm{th}$ component of the surface density
through 
% and $\sigma_m$ is the complex frequency
%of the $m \mathrm{th}$ mode. Since $\sigma_m$ is a value dependent on
%$r$, a more useful diagnostic is the mode growth rate defined as  
\begin{align}\label{growth}
  \frac{d \langle|\Sigma_m|\rangle_r }{dt}= \sigma \langle|\Sigma_m|\rangle_r 
\end{align}
where %$\gamma$ is the real part of the $\sigma_m$ and 
$\langle|\cdot|\rangle_r$ denotes the average of the absolute value
over a radial region of interest. By using Eq.~\ref{growth} the growth
rates of the unstable modes can be found from successive spatial
Fourier transforms over an appropriate period of time. 

\subsubsection{Rossby number}
The Rossby number
\begin{align}
  Ro = \frac{{\bf \hat{z}} \cdot \nabla \times \bv - \langle
    {\bf\hat{z}} \cdot \nabla \times \bv
    \rangle_{\phi}}{2\langle\Omega\rangle_{\phi}},  
\end{align}
is a dimensionless measure of relative vorticity. 
 Here $\langle \cdot \rangle_{\phi}$ denotes an azimuthal
average. Values of $Ro<0$ correspond to anti-cyclonic rotation with
respect to the background shear and thus can be used to identify
vortices and quantify its intensity. 
