\section{Disc-planet models}\label{model}
The system is a two-dimensional non-self-gravitating gas disc orbiting
a central star of mass $M_*$.  
%self-gravitating gas disc of mass
%$M_d$ rotating about a central star of mass $M_*$. 
Cylindrical
co-ordinates $(r,\phi,z)$ centred on the star is adopted. The frame is   
non-rotating. Computational units are such that 
$G=M_*=\mathcal{R}=\mu=1$ where $G$ is the gravitational constant,
$\mathcal{R}$ is the gas constant and $\mu$ is the mean molecular
weight. %% \cite{lin12}

The disc is governed by the standard fluid equations  
\begin{align}\label{3d_gov_eq}
  &\frac{\p\Sigma}{\p t}+\nabla\cdot(\Sigma \bv)=0, \\
  & \frac{\p\bv}{\p t}+\bv\cdot\nabla\bv= -\frac{1}{\rho}\nabla p 
  - \nabla{\Phi} + \bm{f}_\nu,\\
  & \frac{\p e}{\p t} + \nabla\cdot(e\bv) = -p\nabla\cdot\bv +
  \mathcal{H} - \mathcal{C}, %\\
%  &\nabla^2\Phi_d = 4\pi G \Sigma \delta(z),\label{poisson}
\end{align}
where $\Sigma$ is the surface density, $\bv = (v_r,v_\phi)$ the fluid
velocity, $p$ is the pressure, $e=p/(\gamma-1)$ is the energy
density and the adiabatic index $\gamma=1.4$ is assumed constant. 

The total potential $\Phi$ includes the stellar potential, planet potential
(described below)%, and the disc potential $\Phi_d$ which satisfies the
%Poisson equation \ref{poisson}. 
and indirect potentials are also included
to account for the non-inertial reference frame. In the momentum
equations, $\bm{f}_\nu$ represent viscous forces whose magnitude is 
characterised by a constant kinematic viscosity parameter
$\nu$. However, we will be considering effectively inviscid discs by
adopting small values of $\nu$ .  

\subsection{Heating and cooling}
In the energy equation, the heating term $\mathcal{H}$ is defined as 
\begin{align}
  \mathcal{H} \equiv Q^+ - Q^+_i\frac{\Sigma}{\Sigma_i}, 
\end{align}
where $Q^+$ represents viscous heating and subscript $i$ denotes
evaluation at $t=0$. The cooling term $\mathcal{C}$ is defined as
\begin{align}
  \mathcal{C} \equiv \frac{1}{t_c}\left(e -
  e_i\frac{\Sigma}{\Sigma_i}\right),  
\end{align}
where $t_c = \beta\Omega_k^{-1}$ is the cooling time,
$\Omega_k=\sqrt{GM/r^3}$ is the Keplerian frequency and $\beta$ is an
input parameter. Although idealised, this cooling prescription allows one 
to explore the full range of thermodynamic response of the disc in a 
systematic way. 

Note that the energy source terms have been chosen to be absent 
at $t=0$. This allows the disc to be initialised in a steady state
energy, provided the flow is axisymmetric and $v_r$ is negligible compared to $v_\phi$.  
% If the disc is axisymmetric, steady and with only azimuthal flow, then the 
%energy equation is trivially satisfied. 
%This allows one to specify equilibrium discs (see below)
%in the same way as in the case where the energy equation is replaced by a fixed temperature 
%profile (i.e. $c_\mathrm{iso}^2\equiv p/\Sigma$ is prescribed). 
The energy source term attempts to restore the initial energy density (and
therefore temperature) profile. In practice, this is a cooling term at
the gap edge because disc-planet interaction leads to heating.  


\subsection{Disc model and initial condition}
The disc occupies $r\in[r_\mathrm{in}, r_\mathrm{out}]$ with $\rout =
40\rin$ and $\phi\in[0,2\pi]$. The initial disc is axisymmetric with
surface density profile  
 
\begin{align}\label{initial_density}
   \Sigma_i(r) = \Sigma_0\left(\frac{r}{r_\mathrm{in}}\right)^{-s}
    \left[1 - \sqrt{\frac{r_\mathrm{in}}{r + H_i(\rin)}}\,\right] 
\end{align}
where the power-law index $s=2$, $H(r) = \ciso\Omega_k $ defines the disc scale-height 
where $\ciso=\sqrt{p/\Sigma}$ isothermal sound-speed. The disc aspect-ratio is defined as $h\equiv H/r$ and initially
$h=0.05$. For a non-self-gravitating disc, the surface density scale
$\Sigma_0$ is arbitrary. 

%The surface density scale $\Sigma_0$ is set by  
%\begin{align}
%\Sigma_0 = \frac{h_i\rout\Omega_k^2(\rout)}{\pi G Q_o},
%\end{align}
%where $Q_o$ is the thin-disc Keplerian Toomre parameter at the outer boundary initially. 

The initial azimuthal velocity $v_{\phi i}$ is set by centrifugal balance with
pressure, stellar gravity and disc gravity. The initial radial velocity is $v_{r i}=3\nu/r$ 
with $\nu = \hat{\nu}\rin^2\Omega_k(\rin)$. For parameter values employed in this work, $v_{\phi i}\simeq r\Omega_k$ 
and $\hat{\nu}\lesssim O(10^{-5})$, so that $|v_{r i}/v_{\phi i}|\ll1$ and the initial 
flow is effectively only in the azimuthal direction. 


\subsection{Planet potential}\label{planet_config}
The planet potential is given by 
\begin{align}
\Phi_p = -\frac{GM_p}{\sqrt{|\bm{r} - \bm{r}_p|^2 + \epsilon_p^2}},
\end{align}
where $M_p$ is the planet mass and we fix $q\equiv M_p/M_*=10^{-3}$
throughout this work. This corresponds to a Jupiter-mass planet if $M_*=M_{\sun}$. 
The planet's position in the disc 
 $\bm{r}_p=(r_p,\phi_p)$  and $\epsilon_p=0.5r_h$ is a softening
length with $r_h=(q/3)^{1/3}r_p$ being the Hill radius.  
%Hill radius chosen so eps doesn't change with time (scale-height does)
The planet is held on a fixed circular orbit with $ r_p = 10\rin$. This also
defines the time unit $P_0\equiv 2\pi/\Omega_k(r_p)$ used to describe results. 
%In this study only one planet mass is considered, with $q=10^{-3}$. 
%The initial orbital radius is 


\section{Numerical experiments}\label{method}
The disc-planet system is evolved using the 
\texttt{FARGO-ADSG} code \citep{baruteau08, baruteau08b}. This is a modified version 
of the original \texttt{FARGO} code \citep{masset00a} to include the energy 
equation. The code employs finite-difference scheme similar 
to the \texttt{ZEUS} code \citep{stone92}, but with a modified azimuthal transport 
algorithm to circumvent the time-step restriction set by the inner disc boundary. 
The disc is divided into $(N_r,N_\phi)$ zones in the radial and azimuthal directions, 
respectively. The grid spacing is logarithmic in radius and uniform in azimuth.

\subsection{Parametrized cooling}
In this work we only vary one control parameter: the cooling
time. 
%The disc parameters are $Q_o$, which characterizes the strength of self-gravity; 
%and $\beta$ which sets the cooling time. Two levels of self-gravity is considered: 
%a light disc with $Q_o=8$ and a heavy disc with $Q_o=1.5$. 
The cooling parameter $\beta$ is chosen indirectly  through the parameter
$\tilde{\beta}$ such that 

\begin{align}
  t_c(r_p+x_s) = \beta\Omega_k^{-1}(r_p+x_s) = \tilde{\beta} t_{\mathrm{lib}}(r_p+x_s), 
\end{align}
where $x_s$ is the distance from the planet, and $t_\mathrm{lib}$ is
the time interval between successive encounters of a fluid element and
the planet's azimuth. Assuming Keplerian orbital frequencies and $x_s\ll r_p$
gives $t_\mathrm{lib}\simeq 4\pi r_p/(3\Omega_{kp} x_s)$, where
$\Omega_{kp} = \Omega_k(r_p)$. Therefore   
\begin{align}\label{betatilde}
  \beta = \tilde{\beta} \frac{4\pi r_p}{3x_s} \left(1  - \frac{3x_s}{2r_p}\right), 
\end{align}
where $x_s\ll r_p$ was used again. We use $x_s = 2r_h$ in
Eq. \ref{betatilde}. For a planet mass with $q=10^{-3}$,
Eq. \ref{betatilde} then gives $\beta \simeq 23.9\tilde{\beta}$. 


%giant planet gap is 2.5rh
%2rh is CR of G.E.I. 

%\subsection{Simulation procedure}
%The disc is evolved without a planet for $t\in[0,20]P_0$. The planet potential is smoothly switched on
%between $t\in[20,30]P_0$.
%%  In a separate set of calculations the planet
%% potential is turned off after $t>t_\mathrm{off}$, immediately  
%% afterwhich the surface density, energy density and velocity fields are
%% azimuthally averaged and then subject to random surface density  
%% perturbations of magnitude $10^{-2}$.

\subsection{Diagonistic measures}

\subsubsection{Generalized potential vorticity}

The generalized potential vorticity (PV, or vortensity) is defined as
\begin{align}
  \tilde{\eta} = \frac{\kappa^2}{2\Omega\Sigma}\times S^{-2/\gamma}, 
\end{align}
where $\kappa^2 = r^{-3}\partial_r(r^4\Omega^2)$ is the square of the
epicycle frequency, $\Omega=v_\phi/r$ is the angular speed, and
$S\equiv p/\Sigma^\gamma$ is the entropy. The first factor is the
usual potential vorticity. 

The generalized PV appears in the description of the linear stability
of radially-structured adiabatic discs \citep{lovelace99,li00}, where
the authors show an extremum in $\teta$ may lead to a dynamical
instability. In a barotropic disc, the entropy factor is absent
and the important quantity is the PV. 

\subsubsection{Fourier modes}

\subsubsection{Rossby number}

%{\bf {MKL.} describe vortensity and generalized vortensity, and its
%  relevance to stability. fourier mode amplitude definition and how
%  growth rates are obtained. rossby number. density
%  perturbations. (see lin 2014)
%}

%% According to previous 2D studies, the key quantitiy in describing the
%% stability of radially structured non-barotropic discs is the
%% \emph{generalized vortensity} (GV): 
%% \begin{align}
%%   \tilde{\eta}\equiv \left(\frac{\kappa^2}{2\Omega\Sigma}\right)S^{-2/\gamma}, 
%% \end{align}
%% where $S \equiv p/\Sigma^\gamma$ is defined to be the entropy. The
%% quantity in brackets, $\eta\equiv \kappa^2/2\Omega\Sigma$, is the
%% usual vortensity or potential vorticity (PV). 
