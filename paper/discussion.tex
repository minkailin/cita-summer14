\section{Additional results analysis}\label{additional}
In this section we examine some secondary quantities derived
from the hydrodynamic simulations above. To keep this discussion
concise, we will use selected simulations from above for illustration.   

\subsection{Three-dimensionality}
A simple measure of three-dimensionality of the flow is to compare vertical 
to horizontal motion. Since we are interested in non-axisymmetric
perturbations to the gap edge, we first Fourier transform the
meridional momentum densities 

\begin{align}
  (v_{Rm}, v_{zm} ) \equiv\int_0^{2\pi}\rho\times(u_R,
  u_z)\exp{(-\ii m\phi)}d\phi. 
\end{align}
We define the three dimensionality as $\Theta_m(z/H)$, where 
\begin{align}
  \Theta_m^2 \equiv \frac{\avg{|v_{zm}|^2}}{\avg{|v_{zm}|^2}
    +\avg{|v_{Rm}|^2}},  
\end{align}
and $\avg{\cdot}$ denotes a radial average. 
Admittedly, this is a crude measure, and exact values of $\Theta_m$
varies somewhat with details of the average. However, we have
experimented with different averaging domains and 
found the features described below are robust.  


The top panel in Fig. \ref{compare_vprofiles_3d008} shows $\Theta_m$
for Cases 1--3 at $t=40P_0$. The radial average is taken over
$r-r_p\in[3,7]r_h$. These are all vortex modes (see
Fig. \ref{vortex8_polar_dens} and
Fig. \ref{vortex4_vortex10_overall}). The flow becomes increasingly
three-dimensional away from the midplane but 
$\Theta_m = O(10^{-1})$ is small. In an averaged sense the flow is
mostly horizontal. At the end of the  
simulation for Case 1, an azimuthally extended vortex dominates the
flow, for which we measured $\Theta_1\sim
0.2$---0.3. Thus, although vertical motion can become an appreciable
fraction of horizontal motion, the former never dominates.  

$\Theta_m$ for Cases 4---7 are shown in in bottom panel of
Fig. \ref{compare_vprofiles_3d008}. The radial average is performed
over $r-r_p\in[2,6]r_h$ 
because the global spirals in Cases 6---7 significantly protrude the
gap edge. The snapshot is taken at $t=50P_0$ for Cases 4---5, at
$t=40P_0$ for Case 6 and at $t=30P_0$, so that the
vortices and spirals have comparable over-densities at the
gap edge. It also reflects the fact that spiral modes are more
unstable than vortex modes and develop earlier  
\citep{lin11b}. $\Theta_m \sim 0.2$ is again not particularly
large, but the spiral modes are distinctly more three-dimensional than
vortex modes. This is likely due to additional vertical
acceleration provided by the strong self-gravity in those cases.   


%
% can't do detail comparision between cases because instability have
% different growth rates. different 'stages' in instability

% Case 6 show 3 edge disturbances but this is likely to be a mixture
% of spiral and vortex modes, rather than the m=3 spiral mode. This is
% because 
% we clearly see m=2 spiral in Case 7 (Q_0=1.5), and to see higher m spiral
% modes one should increase the strength of self-gravity (e.g. lin11b
% observe m=3 spirals with Q_0=1.2) 
%

\begin{figure}
  \centering
  \includegraphics[scale=.425,clip=true,trim=0.2cm 1.7cm 0cm
    0cm]{figures/compare_vprofiles_3d008}  
  \includegraphics[scale=.425,clip=true,trim=0.2cm 0cm 0cm
    0.2cm]{figures/compare_vprofiles_3d010}  
  \caption{Three-dimensionality of the non-axisymmetric
    flow near the outer gap edge. Top: Cases 1---3 (vortex modes). 
    Bottom: Cases 4---5 (vortex modes) and Cases 6---7 (spiral
    modes). The azimuthal wavenumber $m$ is chosen to match the number
    of vortices or large-scale spirals observed. 
    \label{compare_vprofiles_3d008}}
\end{figure}

\subsection{Vortensity field}\label{vortensity}
A fundamental distinction between the linear vortex and edge mode
instability is their association with local vortensity minimum and
maximum, respectively. In this section we compare vortensity fields of 
discs with vortex modes (Case 2) and edge modes (Case 7). 
%We use Case 2 
%and Case 7 as examples, respectively. 
More specifically, we examine the relative
perturbation to the vertical component of vortensity,  
\begin{align}
  \Delta\eta_z\equiv \frac{\eta_z - \eta_z(t=0)}{\eta_z(t=0)},
\end{align}
where
\begin{align}
  \eta_z \equiv \frac{\bm{\hat{z}}\cdot\nabla\times\bm{u}}{\rho}. 
\end{align}

Fig. \ref{vortex4_vortex1_vortxy} compares $\Delta\eta_z$ in the
midplane when vortices and spirals develop. For planetary gaps, 
vortensity maxima and minima are both located near the gap edges with 
characteristic separation of the local scale-height. The vortensity
ring at the inner gap edge ($r-r_p\simeq - 2r_h$) remain
well-defined. The vortex instability is associated with the local
vortensity minimum near the outer gap edge --- seen as localised closed
contour lines centred about $r - r_p \sim 4r_h$. The vortensity ring at
$r-r_p\sim +2r_h$ becomes distorted as a \emph{consequence} of
large-scale vortex formation just exterior to it. By contrast, the
edge-spiral mode is associated with the local vortensity
maximum. Their development inherently disrupts the vortensity
rings. This is seen in the right panel as the outer ring is broken up.    
    
\begin{figure}
  \centering
  \includegraphics[scale=.425,clip=true,trim=0cm 0cm 1.9cm 
    0.9cm]{figures/vortex4_vortxy_008}\includegraphics[scale=.425,clip=true,trim=2.3cm    
    0.0cm 0cm 
    0.9cm]{figures/vortex1_vortxy_006}
  \caption{Relative perturbation to the vertical component of
    vortensity at the midplane in a disc with the vortex instability
    (left, Case 2 at $t=40P_0$) and the 
    spiral instability (right, Case 7 at $t=30P_0$). Negative
    perturbations in the region $r-r_p\in[2,6]r_h$ are outlined by white lines. 
    Dotted horizontal lines indicate azimuthal cuts taken in
    Fig. \ref{vortex4_vortex1_vortRZ}.   
    \label{vortex4_vortex1_vortxy}}
\end{figure}


The vertical structures also differ. Fig. \ref{vortex4_vortex1_vortRZ}
compares $\Delta\eta_z$ at azimuths coinciding with 
a vortex or the edge disturbance of the spiral mode. Both
instabilities involve $\Delta\eta_z<0$. It is clear that 
the spiral mode has stronger vertical dependence. Its 
region of $\Delta\eta_z<0$ becomes thinner away from
the midplane. In the vortex case this region remains 
about the same width and  $\Delta\eta_z$ is approximately uniform
within it. 
%This is qualitatively
%consistent with increased (downards) vertical motion. 
While $\mathrm{min}(\Delta\eta_z)$ is of comparable
magnitude, the vortensity ring at $r-r_p=2r_h$is much weaker and
thinner in the spiral case ($\Delta\eta_z$ being a factor $\sim 4$
smaller than the vortex case).   
 

\begin{figure}
  \centering
  \includegraphics[scale=.47,clip=true,trim=0cm 1.32cm .0cm
    .6cm]{figures/vortex4_vortRZ_008}\\\includegraphics[scale=.47,clip=true,trim=0cm 
    0.cm .0cm
    .6cm]{figures/vortex1_vortRZ_006}
  \caption{Relative vertical vortensity perturbation
    associated with the vortex instability (top) and spiral
    instability (bottom). The slices are taken at azimuths shown by
    white dotted lines in Fig. \ref{vortex4_vortex1_vortxy}. Negative
    perturbations in the region $r-r_p\in[2,5.5]r_h$ are outlined by
    white lines. \label{vortex4_vortex1_vortRZ}} 
\end{figure}

\subsection{Disc-planet torques}
The presence of non-axisymmetric disturbances at gap edges is
expected to significantly affect disc-planet torques. It has been 
confirmed in 2D simulations that both vortex and spiral modes lead to
oscillatory torques of either sign \citep{li05,lin11b}. 
%Limited 
%numerical resolution the current 3D calculations prevent accurate torque measurements, 
It this section we measure the disc-on-planet torques in several of the above simulations
to confirm the main features found in 2D. 

We calculate the specific torque acting on the planet due to a mass
element as 
\begin{align}
d\bm{T}(\bm{r}) &=
\frac{\bm{r}_p\times\bm{r}G\rho(\bm{r})d^3\bm{r}}{d_p^3}f(\bm{r},\bm{r}_p),\\
f(\bm{r},\bm{r}_p)&\equiv 1
-\exp{\left(-\frac{1}{2}\left|\frac{\bm{r}-\bm{r}_p}{\epsilon_c
      r_h}\right|^2\right)}.  
\end{align} 
The tapering function $f$ reduce contributions from close to the
planet, thereby reducing numerical artifacts arising from this region
because of the diverging potential and limited resolution. We set  
the parameter $\epsilon_c=1$ so that tapering does not significantly
reduce contributions from the instabilities, since they develop
at $\gtrsim 2r_h$ away from the planet's orbital radius.  

Fig. \ref{torque3} shows the disc-on-planet torques in Case 3 and Case 5, which develop
the $m=5$ and $m=6$ vortex modes, respectively. These plots are qualitatively similar to
2D simulations \citep[e.g.][]{li05}. The torques oscillate on orbital 
time-scales and its instantaneous values can be of either sign. However, 
upon averaging over the simulation we find the total torques are negative in both cases. This
means inwards migration is still favoured. 
%The planet is expected to interact

We extended Case 5 to $t=135P_0$ and find the vortices 
have similar over-densities as at $t=50P_0$. However, 
Fig. \ref{torque3} show the torque oscillation amplitudes decrease
towards the end of Case 5 compared to $t\in[40,80]P_0$.  
At $t=50P_0$ the vortices are located in $r-r_p\in[3.5,5.5]r_h$ but by
$t=135P_0$ they are located in $r-r_p\in[4,6]r_h$. 
Given that $t\in[40,80]P_0$ is only $20P_0$ to $50P_0$ after the planet  
potential has been fully introduced, gap-formation is probably 
ongoing during this time. We expect torque amplitudes to be larger
during  gap-formation since the vortices lie closer to the planet. 
%On the
%other hand, gap-widening in the presence of vortices was also seen in  
%2D simulations on long timescales \citep{lin11a}  
%So there may also be
%contributions from radial redistribution of material due to the
%vortices. 
% This might be
%brought about by the of the density bump at the outer gap edge
%\citep{meheut12b}  


%The original
%density bump at the outer gap edge ($r-r_p\simeq 4r_h$) shifts outward
%between these times. 
%This is unlike in the fixed-orbit 2D simulation by \cite{li05}, where
%the oscillation amplitude does not decrease over several hundred
%orbits. 
%The different setups (especially self-gravity and numerical resolution) 
%prevent a detail comparison 

 
\begin{figure}
  \centering
  \includegraphics[scale=.425,clip=true,trim=0.2cm 1.cm 0cm
    0cm]{figures/vortex10_torque3_tqex}  
  \includegraphics[scale=.425,clip=true,trim=0.2cm 0cm 0cm 0.2cm]{figures/vortex2b_torque3_tqex} 
  \caption{Instantaneous disc-on-planet torques in simulations where the vortex mode develops. 
    Top: Case 3. Bottom: Case 5. Note that Case 5 has been extended to $t=135P_0$. \label{torque3}}
\end{figure}

Next we examine disc-planet torques in the presence of the spiral
modes. The top panel of Fig. \ref{torque3_spiral} shows the
instantaneous disc-on-planet torques. Contributions from the
inner disc ($r < r_p$) and outer disc ($r > r_p$) are plotted
separately for comparison with Fig. 18b in \cite{lin11b}, which is 
similar to the present plot. Large oscillations in the outer torque
due to edge mode spirals cause the total torque to be positive or
negative at a given instant. 

Unlike the vortex modes, Fig. \ref{torque3_spiral} shows that spiral
modes can lead to a positive running-time averaged torques (bottom panel). 
The average torques become more positive with time after spiral modes develop, 
and with increasing self-gravity (which increases the instability strength). 
%The torque is more positive with
%increasing strength of the spiral instability.
This was also observed in high-resolution 2D simulations in
\cite{lin11b}. There it was suggested that the creation of large   
`voids' in between spiral arms decreases the time-averaged density in  
the planet-induced wakes, thereby reducing associated torque
magnitudes. Since the outer planetary wake normally provide a negative
torque, the spiral modes make the total torque more positive. 
The similarity between 2D and 3D results indicate that
outwards migration induced by spiral modes, which was seen in 2D by
\cite{lin11b,lin12b}, will also operate in 3D.      

%It was later shown that the spiral instability can supply material to
%go on horseshoe turns ahead of the planet, which also provide a
%positive torque \citep{lin12b}. This is because the spiral modes 
%are associated with local vortensity maximum which are located just
%inside the gap edge in the co-orbital region. 

%This is because the
%spirals are associated with local vortensity maximum, which is just
%inside the outer gap edge. Thus over-densities can be supplied by the
%instability to go on horseshoe orbits

%Since spiral modes are associated with disturbances at the local
%vortensity maximum just inside the outer gap edge, its influence on
%disc-planet torques is expected to be more si

\begin{figure}
  \centering
  \includegraphics[scale=.425,clip=true,trim=0.2cm 1.825cm 0cm
    0cm]{figures/vortex1_torque2_tqex}  
  \includegraphics[scale=.425,clip=true,trim=0.2cm 0cm 0cm 0.2cm]{figures/torque3_tqex} 
  \caption{Disc-on-planet torques in the presence of spiral modes
    associated with the outer gap edge. Top: total torque (solid),
    torque from the inner disc (dotted) and from the outer disc
    (dashed) in Case 7. Bottom: time-averaged torques in Case 6
    (solid) and Case 7 (dotted).
\label{torque3_spiral}}
\end{figure}
