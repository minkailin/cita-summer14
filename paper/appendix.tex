\section{Modification to vertical structure by self-gravity}\label{vertsg_mod}
We describe a simple procedure to set up the vertical structure of a locally
isothermal, self-gravitating disc. We imagine setting up a non-self-gravitating disc,
then slowly switch on the vertical force due to self-gravity. We expect the midplane
density to increase at the expense of gas density higher in the atmosphere. It is assumed
that the temperature profile remains unchanged. 

Vertical hydrostatic equilibrium between
gas pressure, stellar gravity and self-gravity reads
\begin{align}
c_\mathrm{iso}^2(R)\frac{\p\ln{\rho}}{\p z} = -\frac{\p\Phi_*}{\p z} - \frac{\p\Phi}{\p z}. 
\end{align}
Assuming a smooth radial density profile, we use the plane-parallel atmosphere
approximation for the disc potential, i.e.
\begin{align}
\frac{\p^2\Phi}{\p z^2} = 4\pi G \rho. 
\end{align}
Next, we write the density field as
\begin{align}
\rho(R,z) = \rho_N(R, z) \times \beta(z; R)
\end{align}
where $\rho_N$ is the density field corresponding to the non-self-gravitating disc:
\begin{align}
&c_\mathrm{iso}^2(R)\frac{\p\ln{\rho_N}}{\p z} = -\frac{\p\Phi_*}{\p z},\\
&\rho_N = \frac{\Sigma(R)}{\sqrt{2\pi}H(R)}\exp{(-z^2/2H^2)} \notag\\
&\phantom{\rho_N}\equiv \rho_{N0}(R)\exp{(-z^2/2H^2)},
\end{align}
where $\rho_{N0}=\rho_N(R,z=0)$ is the midplane density. The explicit
expression for $\rho_N$ above assumes a thin disc. The function
$\beta$ describes the modification to the local density in order to be
consistent with self-gravity. By construction, its governing equation
is 
\begin{align}
c_\mathrm{iso}^2(R)\frac{\p^2\ln{\beta}}{\p z^2} = -4\pi G\rho_N\beta. 
\end{align}
Let 
\begin{align}
& \chi  = \ln\beta - z^2/2H^2 ,\\
& \xi= \left(\frac{4\pi G \rho_{N0} }{c_\mathrm{iso}^2}\right)^{1/2} z,
\end{align}
then the governing equation can be written in dimensionless form
\begin{align}
&\frac{\p^2 \chi}{\p\xi^2} = -K -\exp{\chi}\label{vertsg_eqn},\\
& K \equiv \frac{c_\mathrm{iso}^2}{4\pi G \rho_{N0} H^2}. \notag
\end{align}
Note that $K(R)$ is proportional to the local Keplerian Toomre
parameter. Eq. \ref{vertsg_eqn} can be further reduced to a first
order differential equation, but this is unnecessary because we
pursue a numerical solution at the end. Appropriate 
boundary conditions are
\begin{align}
&\chi(z=0)  =  \ln\beta_0, \\
&\left.\frac{\p \chi}{\p \xi} \right|_{z=0} = 0.
\end{align}
$\beta_0(R)$ is the midplane density enhancement. To determine its value, we impose
the surface density before and after modification by self-gravity to
remain the same. Then we require 
\begin{align}\label{beta0_eqn}
&F(\beta_0) \equiv \sqrt{\frac{2K}{\pi}}\int_0^{ n /\sqrt{K} } 
\exp{\chi(\xi;\beta_0)}d\xi - \erf{\left(\frac{n}{\sqrt{2}}\right)} \notag\\
&\phantom{F(\beta_0)} = 0,  
\end{align}
where $n$ is the number of scale-heights of the non-self-gravitating
disc we originally considered. At a given cylindrical radius $R$, we
solve Eq. \ref{beta0_eqn} using Newton-Raphson iteration. Each
iteration involves integrating the governing ODE for $\chi$
(Eq. \ref{vertsg_eqn}). At the end of the iteration, we have $\beta(z;
R)$ and the midplane enhancement $\beta_0(R)$.  

We comment that the procedure outlined above can be extended to
polytropic discs. In this case, there is an additional unknown --- 
the new disc thickness after adjustment by self-gravity and an
additional constraint --- the density should vanish at the new disc
surface. 
