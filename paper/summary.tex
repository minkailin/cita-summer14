% {\bf question: vortex amplitude plot suggest just after linear growth,
%   vortex amplitudes are smaller with increasing beta, but later on in
%   the quasi-steady state, this trend reverses? yes. 
% }
\section{Summary and discussion}\label{summary}
In this paper, we have carried out numerical simulations of
non-isothermal disc-planet interaction.  
Our simulations were customized to examine the effect of a finite 
cooling time on the stability of gaps  
opened by giant planets to the so-called vortex or Rossby wave
instability. To do so, we  
included an energy equation with a cooling term that restores the 
disc temperature to its initial profile on a characteristic timescale
$t_c$. We studied the evolution of the gap stability as a function of 
$t_c$. This is a natural extension to previous studies of on gap
stability, which employ locally or strictly isothermal equations of
state. We considered the inviscid limit which favors the RWI
  \citep{li09,fu14} and avoids complications from viscous
  heating other that shock heating. However, this means that the vortex lifetimes observed in
  our simulations are likely longer than in realistic discs with
  non-zero physical viscosity.    	

We considered two types of numerical experiments. We first used
disc-planet interaction to self-consistently set up gap profiles,
which were then perturbed and evolved without further the influence of
the planet potential. This procedure isolates the effect of cooling on
gap stability through the set up of the initial gap profile. We find
that as the cooling time $t_c$ is increased, the gaps became more
stable, with lower growth rates of non-axisymmetric modes and the
dominant azimuthal wavenumber also decreases. This is consistent with
the notion that increasing $t_c$ leads to higher temperatures or
equivalently the disc aspect ratio $h$,
which opposes gap-opening by the planet. This means that the gaps
opened by the planet in a disc with longer $t_c$ are smoother and
therefore  more stable to the RWI. %generalized vortensity minima less
                                %pronounced    

In the second set of calculations, we included the planet potential
throughout the simulations  and examined the long-term evolution of
the gap-edge vortex that develops from the RWI. The vortex reaches 
a quasi-steady state lasting $O(10^3)$ orbits. Unlike the `planet-off'
simulations, in which vortices decay after linear growth and merging,
we find that with the planet potential kept on, the vortex amplitude
grows during this quasi-steady state, during which no vortex migration is
  observed, until it begins to induce
shocks, after which the vortex amplitude begins to decay.   

{\bf For our main simulations with $\tilde{\beta}\geq 0.1$}, the duration of the quasi-steady state increases with
increasing cooling timescales until a critical value, beyond which this
quasi-steady state shortens again. {\bf We find the timescale for the vortex to
decay after reaching maximum amplitude can be long for small
$\tilde{\beta}$, which contributes to a long overall vortex lifetime
with rapid cooling.} We do observe vortex migration during {\bf its}
decay, which may influence this decay timescale. 
  
% The sum of
% these two timescales results in a double-peaked vortex lifetime as a
% function of cooling timescale.

We suggest a non-monotonic dependence of the quasi-steady state {\bf
on the cooling timescale $\tilde{\beta}$} can
be attributed to the time required for the vortex to grow to 
sufficient amplitude to induce shocks in the surrounding fluid,
thereby losing energy and also smooth out the gap edge.   

For short cooling timescales, the planet is able to open a
deeper gap which favours the RWI, leading to stronger 
vortices. For long cooling timescales, we find the vortex
grows faster during the quasi-steady state. In accordance with
previous stability calculations \citep{li00}, we 
suggest the latter is due to the RWI being favoured with increasing
disc temperature, and that this effect overcomes weaker
gap-opening for sufficiently long cooling times. 
These competing factors imply
for both short and long cooling timescales, the vortex reaches
its maximum amplitude, shock, and begins to decay, sooner than
intermediate cooling timescales. 

{(\bf However, for very rapid cooling, e.g. $\tilde{\beta}=0.01$, the quasi-steady state is also
  quite long. This suggests that the above effects themselves do not
  have a simple dependence on the cooling timescale when considering
  $\tilde{\beta}\to0$ and/or that other factors become important in
  this limit. This should be investigated in future works.)} 

%{\bf, for which the quasi-steady state is 
%the longest.}      
%decay timescale depends on cooling time, but there seems to be a
%monotonic trend: longer cooling times, decay timescale is shorter, so
%this  
%doesn't contribute to non-monotonic trend  

We remark that a non-monotonic dependence of the vortex lifetime was
also reported by \cite{fu14}, who performed {\bf locally} isothermal disc-planet
simulations with different values of the   
disc aspect ratio. In their simulations the optimum aspect ratio is
$h=0.06$. In our simulations, $h$ is a dynamical
variable, but by analyzing the region where the vortex is located
($r-r_p\in[2,10]r_h$), % {\bf define this region}
we find for a dimensionless cooing timescale of $\tilde\beta=2.5$, which has
the longest vortex lifetime {\bf in the presence of moderate cooling}, that  
$h\approx0.058$ on average.  
%{\bf if new definition of vortex lifetime gives other $\tilde{\beta}$ values as the maximum lifetime, then need to update this value of $h$}. 
Our result is consistent with \cite{fu14}.

% we don't know how fu defined vortex lifetime

\subsection{Caveats and outlooks}
There are several outstanding issues that needs to be addressed
in future work: 

\emph{Convergence.} Although lower resolution simulations performed in
the early stages of this project gave similar results  
(most importantly, the non-monotonic dependence of vortex lifetimes on
the cooling timescale), we did find the lower resolution 
typically yield longer vortex lifetimes than that reported in this
paper. This could be due to weaker RWI with low resolution.  
It will be necessary to perform even higher resolution  simulations
in order to obtain quantitatively converged vortex lifetimes.  


  \emph{Orbital migration.} We have held the planet fixed on a circular
  orbit. However, gap-edge vortices are known to exert
  significant, oscillatory torques on the planet \citep{li09}
  which can lead to complex orbital migration. This will likely affect
  vortex lifetimes as it may alter the planet-vortex separation, as
  well as leading to direct vortex-planet interactions
  \citep{lin10,ataiee14}. Thus, 
  future simulations should allow the planet to freely
  migrate. Similarly, the role of vortex migration on its lifetime
  should be clarified. 


  \emph{Cooling model.} Our prescription for the disc 
  heating/cooling is convenient to probe the full range of
  thermodynamic response of the disc. However, in order to calculate 
  vortex lifetimes in actual 
  protoplanetary discs, an improved thermodynamics treatment,
  e.g. radiative cooling based on realistic disc temperature, density,
  opacity models etc., should be used in future work.   

\emph{Self-gravity.} We have ignored disc self-gravity in this
study. Based on linear calculations, \cite{lovelace13} concluded
self-gravity to be important for the RWI when the Toomre parameter $Q<O(1/h)$, or
$Q\lesssim 20$ for $h\sim0.05$, as was typically considered in this
work. This suggests that self-gravity may affect vortex lifetimes even
when $Q$ is not small. In particular, given that
we observe vortices can reach significant over-densities (up to almost
an order of magnitude), it will be important to include disc
self-gravity in the future. %self-gravitational collapse of vortices 
 
\emph{Three-dimensional (3D) effects.} A vortex in a 3D disc may be
subject to secondary instabilities that destroy them
\citep{lesur09,railton14}. This may be an important factor in
determining gap-edge vortex lifetimes in realistic discs. For example,
if these secondary instabilities sets in before the vortex grows to
sufficient amplitude to shock, then the dependence of the vortex
lifetime on the cooling timescale will be its effect through the 3D
instability (as opposed to the effect on the RWI itself, which is a 2D
instability). This problem needs to be clarified with full 3D
disc-planet simulations.  

%3d is challenging because high resolution needed in for 3d instabilities
%However, the role of these instabilities on gap-edge vortices, where disc-planet  
%interaction can maintain the RWI, has not been clarified.  
