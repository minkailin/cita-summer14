\section{Summary and discussion}\label{summary}
We have performed customised numerical simulations of  
three-dimensional self-gravitating discs, in which an embedded 
satellite or planet has opened a gap. We explicitly verified in 
3D the main results on gap stability previously obtained from 
2D calculations \citep{koller03,li05,valborro07,meschiari08}, in
particular those by \cite{lin11a,lin11b}.  

Planetary gaps are potentially unstable because of the existence of 
vortensity extrema generated by planet-induced shocks. Disc-satellite
interaction also occur in other systems such as stars in black hole
accretion discs \citep{kocsis11,mckernan11,baruteau11b}.  Furthermore, 
gaps opened by satellites are just one type of structured disc. 
Other examples 
include dead zone boundaries mentioned in \S\ref{intro} and
transition discs \citep{regaly12}.  

Thus, although we were motivated by previous works on planetary gap 
stability, and hence considered disc-planet systems, we expect our
results to be applicable to discs with radial structure of other    
origin, provided the vortensity profiles involve stationary points and
therefore prone to the same instabilities.    

\subsection{Confirmed 2D results}
We demonstrated the development of the vortex instability
at the outer gap edge opened by a giant planet in 3D 
discs. We began with a non-self-gravitating disc, in which a few vortices
develop then quickly merge. The quasi-steady state is a single
azimuthally extended vortex at the gap edge. This evolution is similar
to the 2D simulation in \cite{valborro07}.  

We also showed the effect of self-gravity on the vortex
instability observed in 2D \citep{lyra09,lin11a} persists in 3D.
We observe more vortices as the strength of self-gravity is increased
by increasing the density scale. These vortices resist merging
on dynamical timescales. In our disc models with $Q_0=3$, the 
multi-vortex configuration lasts until the end of the
simulations.        

As expected from 2D linear theory \citep{lin11b}, 
vortex modes are suppressed in our massive disc models. Instead, a 
global spiral instability develops which is associated with the local
vortensity maximum just inside the outer gap edge. These are distinct
from vortex modes since self-gravity is essential. 

Our limited numerical resolution does not permit accurate disc-planet
torque measurements, but 
the qualitative effect of the vortex and spiral instabilities,
previously studied in 2D, have been reproduced in 3D --- oscillatory
torques of either sign and the tendency for spiral modes to provide on
average a positive torque. 
The similarity to 2D results is not surprising since for  
giant planets $r_h\gtrsim H$, so the razor-thin disc approximation is
expected to be valid as far as disc-planet interaction is
concerned. Furthermore, vertical self-gravity increases the midplane
density while reducing that in the atmosphere (Appendix
\ref{vertsg_mod}), so that given a fixed temperature profile the 2D
approximation is even better for self-gravitating discs. 





 It is worth mentioning here that in shearing sheet
  simulations,  \cite{mamatsashvili09} found 
  self-gravity to favour vortices of smaller scale. They initialized
  a local patch of an unstructured disc with random velocity perturbations.
  Their gravito-turbulent discs are dominated by small vortices limited by
  the local Jeans length (which is smaller than the scale-height). Without self-gravity, 
  they merge to form larger vortices. 
  These observations are similar to the above results for the vortex mode.  
  However, the setups are quite different 
  as we consider radially structured, laminar global discs. Our
  large-scale vortices develop from a linear instability and have horizontal sizes
  of a few scale-heights. Thus, confirmation of the above results in 3D is only valid
  for the edge instabilities considered in this paper.   



%torques (expected because giant planets are 'thicker' than disc)

\subsection{Effects of 3D}
In our simulations the dominant three-dimensional effect is
vertical self-gravity on the density field. In the
non-self-gravitating limit, the relative density perturbation
associated with a vortex is columnar with weak vertical dependence. This is
consistent with with 3D linear and nonlinear simulations
\citep{meheut10, meheut12a,  meheut12b, lin12}.       

As the strength of self-gravity is increased, vortices become 
more vertically stratified --- they are condensed towards the
midplane. For moderately self-gravitating discs, the vortex
midplane density enhancement can be twice that near the upper
boundary.  The spiral modes display significant vertical structure
near the gap edge, while the density waves they launch in the outer
disc is columnar.  The latter is probably due to the chosen equation of state (see below).


The effect of vertical self-gravity on the vortex mode is seen even in
our least massive disc model with $Q_0=8$. One can consider an
initially smooth disc which is justified to be
non-self-gravitating. However, this approximation may become less good
with the creation of a vortensity minimum, because it
can also be a local minimum in the Toomre $Q$
(Eq. \ref{toomreq_vortensity}).  That is, the Toomre parameter is
decreased with the development of local radial structure (such as a density bump). 
The non-self-gravitating approximation worsens further when the vortex
instability associated with $\mathrm{min}(\eta)$ develops, because the   
vortices are regions of enhanced density, especially if they merge
into a single large vortex. Thus, the
non-self-gravitating approximation is not guaranteed to hold 
in the perturbed state even if it does in the initial disc.  


It is interesting to note that linear calculations of the vortex
instability in vertically isothermal , non-self-gravitating discs 
show that the vertical velocity vanishes near the vortex
centroid  \citep{meheut12a,lin12}, but this is not observed in
nonlinear calculations \citep[][ and in the present
  simulations]{meheut12a}.  This contrasts to their anti-cyclonic
horizontal flow, which can be computed in linear theory and seen in
hydrodynamic simulations \citep{li00,li01}.  This suggests that vertical 
motions in this case may be associated with secondary processes
\citep[e.g][]{goodman87}. 

Although we observe somewhat complicated vertical flow for both
vortex and spiral instabilities, the vertical Mach number is
at most a few per cent in magnitude. Also, the magnitude of
vertical motion is at most $\sim 20\%$  of the radial motion on
average. This suggests that the disturbances at the gap edge is 
roughly two-dimensional in the present disc models. This would be 
consistent with early studies which find instabilities associated with
co-rotation singularities are two-dimensional
\citep{papaloizou85,goldreich86}.  Recent 3D linear 
  calculations also find that, even with a vertical temperature gradient, 
  the vortex mode (without self-gravity) is
  largely 2D near the vortensity minimum \citep{lin12}.

\subsection{Caveats and outlooks}\label{caveats}

In order to provide 3D examples of previous 2D results, a range of 
disc models had to be simulated, each for many dynamical 
timescales with full self-gravity. To maintain reasonable
computational cost, numerical resolution in the $r\phi$ plane is much
reduced compared to razor-thin disc simulations ($\sim  4$---6 cells
per $H$ compared to $\sim 16$ in 2D).  Despite this, our
plots in the $r\phi$ plane closely resemble those obtained from 2D
simulations.  

On the other hand, the low resolutions adopted here are unlikely to
capture elliptic instabilities which may destroy 
vortices in 3D \citep{lesur09b,lesur10}. 
This might not be a serious
issue when the condition for the vortex instability is maintained 
(by a planet in the present context). Also, the vortex 
grows on dynamical timescales whereas the elliptic instability takes
much longer \citep{lesur09b}. The initial development of vortices is not expected to be 
suppressed by the elliptic instability \citep[as found
  by][]{meheut12a}.   

The vortex instability produces smaller vortices with increasing
self-gravity. According to \cite{lesur09b}, vortices with small
aspect-ratios ($\lesssim 4$) are strongly unstable in 3D, but note that their  
model is a local patch of a smooth disc without self-gravity, 
whereas we considered a gap edge in a global self-gravitating disc. 

Inclusion of self-gravity may change the stability properties of
vortices. In particular, we found that a vortex can flatten somewhat under its own
weight. \cite{lithwick09} suggested vertical gravity helps to 
stabilise vortices in 3D. Vertical \emph{self}-gravity 
can enhance this effect.  \citeauthor{lithwick09} 
found in local 3D simulations that `tall' vortices are unstable 
whereas `short' vortices survive as in 2D \citep{godon99}. 
We may expect the more stratified vortices formed in self-gravitating 
discs to be more stable than those in non-self-gravitating discs \citep{barranco05}. 
The elliptic instability is also weakened by stratification \citep{lesur09b}, 
so vertical self-gravity may also be stabilising in this respect. 

%Improved numerical models, including the energy equation, should be 
%simulated to explore these speculations. 
% 
%vortex instability on dynmical timescales. elliptic instability is
%weaker growth 

 
We have adopted the locally isothermal EOS for simplicity and direct
comparison with previous 2D results. This EOS limits the 3D 
structure of density waves compared to thermally 
stratified discs \citep[e.g][]{lin90,lubow98,ogilvie99}, which can 
cause refraction of waves out of the midplane.  
The locally isothermal EOS represents the limit of efficient cooling
\citep{boss98}. This might apply in optically thin
regions of a disk\footnote{For example, \cite{cossins10} find optical
depths $\tau < 0.2$ beyond $\sim100\mathrm{AU}$ in their models of protoplanetary
discs.}, but is violated if high densities develop, such as 
self-gravitating clumps \citep{pickett00}. Our edge instabilities only 
reach moderate over-densities. Nevertheless, enhanced
vertical stratification of the edge disturbances observed in our
simulations are likely exaggerated by the EOS. Clearly, it is necessary to
extend models of edge instabilities in self-gravitating discs to
include an appropriate energy equation. 
                                                            

Another important issue is vertical boundary conditions. We simulated
a thin disc and imposed a reflecting upper boundary to prevent mass
loss from above. This setup  may enhance the two-dimensionality
of the problem. The vortex instability tends to involve the 
entire column of fluid, especially in the weak self-gravity regime. It
is therefore a global instability in $z$. We suspect the spiral mode
is less affected by vertical boundary conditions because the
instability tends to concentrate material at the midplane. Future
simulations will consider varying the vertical domain size and upper
disc boundary conditions. 

Our torque measurements indicate migration will be significantly
affected by the vortex and spiral instabilities. Preliminary 3D
simulations with a freely migrating planet have been performed. We
recover the vortex-planet scattering and the spiral-induced outward
migration described by \cite{lin10,lin12b}.   
The disc-planet torque is determined by the density field, and
the above instabilities have density perturbations that either
have weak vertical dependence or concentrated at the midplane. Thus, we
believe that at present, 2D simulations are more advantageous for studies
focusing on migration, because high resolution is feasible and needed.   
